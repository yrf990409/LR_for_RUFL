\setlength{\baselineskip}{16pt}
\chapter{附录A}
\label{cha:app1}
%\section*{附录标题}
%\zihao{5}
%[内容为五号宋体。] 附录是作为论文主体的补充项目,并不是必须的。
%论文的附录依序用大写正体英文字母A、B、C……编序号,如:附录A。
%\vspace{0.75cm}
\begin{center}
\zihao{3}
\textbf{ILS-LR算法源码}
\end{center}

\indent
\zihao{-4}
该代码包含算法的主程序及所有函数,
算法所需的数据请见正文中的参考文献。
此外,作者的Github开源仓库\footnote{https:\slash \slash github.com\slash yrf990409\slash LR\_for\_RUFL}
提供了同样的代码、数据以及环境配置方法,
以方便感兴趣的读者复现。
本附录的第一部分是LR-ILS算法的Matlab算法,
第二部分是Gurobi建模Python代码。

\subsection*{LR-ILS算法源码(Matlab)}
\vspace{-20pt}
\small{
    \begin{lstlisting}[language=Matlab,
        breaklines=true, % 自动换行
        frame = leftline,
        linewidth = \textwidth,
        texcl = false,
        columns=flexible,
        lineskip=1pt,
        keywordstyle=\color{blue!90}\bfseries, %代码关键字的颜色为蓝色,粗体
        numbers=left,%左侧显示行号
        numberstyle=\small, %行号字体用小号
        showspaces=false, %
        fontadjust,
        basicstyle = \fontspec{Times New Roman},
        basicstyle = \songti
        ]
    % 拉格朗日松弛求解可靠设施选址问题(RUFL)
    % 编码方式: UTF-8
    % 操作系统: Windows 10/11 & MacOs 13
    % 运行平台:
    % Matlab R2022a/b 
    % 工具箱 Matlab Coder & Parallel Computing
    % Gurobi 9.5.2
    % Python 3.10

    % 清除
    clc
    clear all
    clear classes
    close all 
    diary off
    diary 'LR.log'
    disp(['程序开始: ', datestr(now)])

    %% 参数控制
    lr_para = struct();             % LR 算法的参数
    lr_para.alpha       = 2;        % 步长参数初始值
    lr_para.alpha_min   = 0.0001;   % 步长参数最小值
    lr_para.theta_lr    = 1.05;      % 步长比例系数
    lr_para.kappa_lb    = 10;       % 下界连续不变次数
    lr_para.eta_lr      = 3000;     % 迭代次数
    lr_para.tau_lim     = 1000;     % 优化时长
    lr_para.xi          = 0.01;     % 接受gap
    lr_para.theta_sa    = 1.2;      % 模拟退火初始温度参数
    lr_para.T_lim       = 0.0001;   % 模拟退火最低温度
    lr_para.kappa_ub    = 100;      % 触发ILS上界连续不变次数 (200)
    lr_para.kappa_ubdfs = 500;      % 触发上界DFS搜索(delete)
    lr_para.eta_ils     = 10;       % ILS迭代次数
    lr_para.grb_model   = 1;        % 使用Gurobi建模 取值 0 1
    lr_para.grb_ub      = 1;        % 使用Gurobi获取上界 取值0 1
    lr_para.dfs_gap     = 0.2;      % gap小于此值才启动DFS
    lr_para.print       = true;     % 解池的大小

    %% 案例参数
    % lr_case = struct();
    % path = './data/SnyderData/49nodes/';
    lr_case.data = data_reader(path);
    lr_case.rho = prob_rho;                                     % 损坏概率控制参数
    lr_case.q = lr_case.rho * exp(-lr_case.data.fix/200000);    % 损坏概率
    lr_case.q(1) = 1;                                           % 虚拟设施的损坏概率
    lr_case.max_try = out_index;                                % 最大尝试次数(R)
    lr_case.cus_num = length(lr_case.data.dmd);                 % 客户数量
    lr_case.node_num = size(lr_case.data.price,1);              % (虚拟 实体 客户)的总数
    lr_case.fac_num = lr_case.node_num - lr_case.cus_num - 1;   % 设施总数
    lr_case.bar_J = (0:lr_case.fac_num)+1;                      % 设施集合拓展集
    lr_case.I = (lr_case.fac_num+1 : lr_case.node_num-1)+1;     % 客户集合
    lr_case.mu = zeros(lr_case.cus_num, length(lr_case.bar_J)); % 生成初始乘子
    for i = 1:lr_case.cus_num
        cus_ind = lr_case.I(i)-lr_case.I(1)+1;
        for j = lr_case.bar_J(2:end)
            lr_case.mu(:,j) = (lr_case.data.dmd(cus_ind) * lr_case.data.price(cus_ind,j) + ...
                lr_case.data.fix(j)) / length(lr_case.data.fix);
        end
    end

    %% 优化
    lr_result = lr_ils(lr_para, lr_case);

    %% 结果处理
    % draw_fig(lr_result)

    % 保存
    file_name = ['结果', ...
                num2str(lr_case.node_num), '-' , ... % 点数
                num2str(lr_case.rho),      '-' , ... % rho取值
                num2str(lr_case.max_try),  '-' , ... % 最大试错次数
                '.mat'];
    save(file_name,'lr_result')

    % fprintf('rho \t 最大试错次数 \t 上界 \t 下界 \t 求解时间\text{\\n}')
    fprintf('%d \t %.2f \t %d \t %.2f \t %.2f \t %.2f \n', lr_case.node_num, lr_case.rho, lr_case.max_try, lr_result.bst_ub, lr_result.bst_lb, lr_result.time)
    diary off

    function data = data_reader(f_path)
    %DATA_READER 读取数据
    %   data包含三个数据
    %% 数据读取
    % disp(['导入数据,当前路径: ', f_path])
    data.price = load([f_path, 'cost.csv']); % 价格矩阵
    data.dmd = load([f_path, 'dmd.csv']);    % 客户需求
    data.fix = load([f_path, 'fc.csv']);     % 固定成本
    % disp('导入完成!')
    % disp('---------------------------------')

    % 验证数据
    validateattributes(data.price, {'double'}, {'size', [length(data.dmd)+length(data.fix), length(data.fix)]})

    end

    function [ub,best_r] = lb_dfs(r, best_r, ub, cur_cost, cur_prob, fac, cus_dmd, ...
        cost_mat, R, pi, probDisr, cus_mu)
    % global x
    % x = x+1
    if length(r) <= R+1 % 没有到达设施数量的上界
        if r(end) == 1 % 如果结尾是虚拟,那么获得了一个完整路径
            tempCost = cur_cost;
            if tempCost < ub % 小于上界 则更新
                ub = tempCost;
                best_r = r;
                return
            end
            return
        else % 否则不是一个完整路径
            if isempty(fac) % 没有设施可以填充了,route最后一个不以虚拟结尾
                r = [r,1]; % route必须是虚拟收尾了

                tempCost = cur_cost + cus_dmd*pi*cur_prob;
                if tempCost < ub % 小于上界 则更新
                    ub = tempCost;
                    best_r = r;
                    return
                end
                return
            else % 还有设施可以填充
                if length(r) == R+1 % 已经填满,放不下了,只差一个虚拟设施
                    r = [r,1]; % route必须是0收尾了
                    tempCost = cur_cost + cus_dmd*pi*cur_prob;
                    if tempCost < ub % 小于上界 则更新
                        ub = tempCost;
                        best_r = r;
                        return
                    end
                    return
                else % 向下继续分支
                    for i = 1:length(fac) % i : index of facilities
                        stackR = r;
                        stackC = cur_cost;
                        stackP = cur_prob;

                        r = [r,fac(i)];
                        tempCost = cur_cost + cost_mat(r(end-1),r(end))*cur_prob*cus_dmd + cus_mu(1,fac(i));

                        if tempCost>ub % 剪枝
                            r = stackR; % 直接恢复不再细分
                        else
                            tempFacy = fac;  % 递归 继续分支
                            tempFacy(tempFacy==fac(i)) = [];
                            cur_prob = cur_prob*probDisr(r(end));
                            [ub,best_r] = lb_dfs(r,best_r,ub,tempCost,cur_prob,tempFacy,cus_dmd,cost_mat,R,pi,probDisr,cus_mu);
                            r = stackR;
                            cur_cost = stackC;
                            cur_prob = stackP;
                        end
                    end
                end
            end
        end
    else
        return
    end
    end
    
    function [trans_cost, plan] = lb_x(lr_case, flag_fast)
    %lb_dfs_x 深度搜索获取每个客户的尝试路径

    % 输入 lr_case 字段
    % I         客户索引 向量
    % bar_J     设施索引 向量
    % data      数据 结构体 字段 price价格矩阵 dmd客户需求 fix固定成本
    % q         每个设施失效概率 向量
    % mu        拉格朗日乘子 矩阵
    % max_try   客户最大尝试次数 整数
    % dfs_falg  是否使用DFS 逻辑值

    % 输出
    % trans_cost    期望运输成本   浮点数矩阵
    % plan          计划          整数矩阵

    %% 初始化
    % 提取
    I = lr_case.I;
    bar_J = lr_case.bar_J;
    data = lr_case.data;
    q = lr_case.q;
    mu = lr_case.mu;
    max_try = lr_case.max_try;
    trans_cost = zeros(length(I),1);    % 记录每个客户的期望运输成本

    %% 贪心路径构建
    [plan, trans_cost] = greedy_build(I, max_try, bar_J, data, mu, q, trans_cost);

    %% DFS改进当前解
    if ~flag_fast % 快速模式不启动dfs
        dmd = data.dmd;
        price = data.price;
        pi = price(2,1);    % 惩罚成本
        parfor j = 1:length(I)
            cus = I(j);
            best_r = plan(j,:); % 当前最优路径
            ub = trans_cost(j); % 当前上界
            cus_dmd = dmd(j); % 客户的需求
            cus_mu = mu(j,:);  % 拉格朗日乘子

            R = coder.ignoreConst(max_try-1);
            coder.varsize('cus', [1 100], [0 1]);
            coder.varsize('best_r');
            coder.varsize('bar_J');
            coder.varsize('price');
            coder.varsize('q');
            coder.varsize('cus_mu');

            [ub, best_r] = lb_dfs(cus, best_r, ub, 0, 1, bar_J, cus_dmd, price, R, pi, q, cus_mu); % 深度优先搜索
            trans_cost(j) = ub; % 更新上界

            if length(best_r) < max_try+1
                plan(j,:) = [best_r, zeros(1, max_try+1-length(best_r))];
            else
                plan(j,:) = best_r; % 记录方案
            end

        end
    end
    end


    function [plan, trans_cost] = greedy_build(I, max_try, bar_J, data, mu, q, trans_cost)
    % 路线构建的贪心算法
    % 在每个路径的最后增加一个算子
    plan = zeros(length(I), max_try+1); % 记录每个客户的方案
    dmd = data.dmd; % 客户的需求
    price = data.price; % 价格矩阵
    for k = 1:length(I)
        cus = I(k);
        % 直接采用构造法生成一个初始路径
        route = zeros(1, max_try+1); % 路径
        route(1) = cus; % 路径的起点是客户
        prob = 1;   % 初始概率是1
        fee = 0;    % 初始费用是0
        fac = bar_J;    % 设施的复制
        cus_dmd = dmd(k); % 客户的需求
        jdg_brk = false;
        
        mu_k = mu(k,:);
        % 每次讲增量成本最小的设施放在路线最后(贪心构建)
        for i = 2:max_try
            add_fee = zeros(1,length(fac));
            for j = 1:length(fac)
                add_fee(j) = prob * cus_dmd * price(route(i-1),fac(j)) ...
                    + mu_k(fac(j));
            end
            [min_add_fee,min_ind] = min(add_fee); % 指向最小的费用
            route(i) = fac(min_ind); % 路线更新
            prob = prob * q(route(i)); % 概率更新
            fee = fee + min_add_fee; % 费用更新
            if route(i) == bar_J(1)
                jdg_brk = true;
                break % 如果虚拟设施被添加到路径中则退出
            end
            fac(min_ind) = []; % 否则去除已经添加的设施
        end

        if jdg_brk % 提前退出
            plan(k,:) = route; % 记录路径
            trans_cost(k) = fee; % 记录成本
        else
            route(route==0) = []; % 去除路径中的多余0
            trans_cost(k) = fee + cus_dmd * prob * price(route(end),1); % 直接计算成本

            formated_route = [route, 1, zeros(1, max_try-length(route))]; % 记录路径
            plan(k,:) = formated_route; % 记录路径
        end
    end
    end


    function [cost_for_fac, location] = lb_y(lr_case)
    %LB_Y获取y变量的下界
    % 给定一个乘子,返回y的最佳选址location以及设施建设成本cost_for_fac

    % 初始化
    bar_J = lr_case.bar_J;
    data = lr_case.data;
    mu = lr_case.mu;

    location = false(1,length(bar_J));     % 选址方案
    cost_for_fac = zeros(1,length(bar_J)); % 下界Sub_1问题的目标函数

    for j = 2:length(bar_J)
        fac = bar_J(j);
        sum_mu = sum(mu(:,fac)); 
        jdg = data.fix(fac) - sum_mu; % 判别

        if jdg > 0 % 判别数大于0 不选这个位置
            location(fac) = 0;
        else
            location(fac) = 1; % 小于等于0 选
            cost_for_fac(fac) = data.fix(fac) - sum_mu;
        end

    end

    end

    function lr_result = lr_ils(lr_para, lr_case)
    % LR_ILS 拉格朗日松弛-局部迭代搜索算法
    % 传入 案例参数lr_case 以及算法参数lr_para
    % 传出 lr_result结果

    %% 初始化
    tic
    % 记录 recorder
    rec_lb  = zeros(lr_para.eta_lr, 1);     % 下界记录 record
    rec_ub  = zeros(lr_para.eta_lr, 1);     % 上界记录
    rec_ils = false(lr_para.eta_lr, 1);     % ILS发挥作用记录

    % 计数器 count
    cnt_step = 0;   % 下界连续不上升计数
    cnt_ils  = 0;   % 上界连续不下降计数
    cnt_ub   = 0;   % 最佳上界保持的迭代次数
    cnt_iter = lr_para.eta_lr;  % 算法真正的迭代次数

    % 最佳解 best_solution
    bst_loc  = false(1, length(lr_case.bar_J));             % 最佳选址方案 location
    bst_sqc  = zeros(length(lr_case.I), lr_case.max_try+1); % 最佳客户序列 sequence
    bst_ub   = inf;     % 最佳上界
    bst_lb   = -inf;    % 最佳下界
    gap = 1;

    % 条件判断 flag
    flag_fast = true;       % 算法快速模式
    flag_modify = false;    % 快速\&慢速切换时强制更正上下界
    flag_ils_continue = false;  % ILS是否继承上一次的结果

    % 邻域
    cur_ub = inf;
    cur_loc = false(1,length(lr_case.bar_J));     % 选址方案

    %% LR-ILS优化
    for iter = 1:lr_para.eta_lr
        % 获取下界
        [lb_val_x, lb_sqc] = lb_x(lr_case, flag_fast);  % 获取Sub_2的值
        [lb_val_y, lb_loc] = lb_y(lr_case);             % 获取Sub_1的值
        lb = sum(lb_val_x) + sum(lb_val_y);

        % 获取上界
        [ub, sqc, ~] = ub_xy(lr_case, lb_loc, flag_fast);

        % ILS搜索上界
        if cnt_ub >= lr_para.kappa_ub && gap > 2*lr_para.xi
            if flag_ils_continue
                [nb_ub, nb_loc, cur_ub, cur_loc] = ub_ils(lr_case, cur_ub, cur_loc, lr_para);
            else
                [nb_ub, nb_loc, cur_ub, cur_loc] = ub_ils(lr_case, bst_ub, bst_loc, lr_para);
            end

            if nb_ub < bst_ub
                % 得到更好的上界
                bst_loc  = nb_loc;  % 选址方案
                bst_ub   = nb_ub;   % 上界 
                rec_ils(iter) = true;
            else
                flag_ils_continue = true;  % ILS继承上一次的结果
            end

            cnt_ub = 0;
        end

        % 记录最佳上界
        if ub <= bst_ub
            % 得到了更好的上界
            bst_loc  = lb_loc;  % 选址方案
            bst_sqc  = sqc;     % 客户序列
            bst_ub   = ub;      % 上界
            % 计数器更新
            cnt_ils = 0;        % 上界连续不下降
            cnt_ub  = 0;        % 最佳上界保持次数
            % 局部迭代搜索更新

        else
            % 计数器更新
            cnt_ils = cnt_ils + 1; % 上界连续不下降
            cnt_ub  = cnt_ub + 1;  % 最佳上界保持次数
        end

        % 记录最佳下界
        if lb >= bst_lb
            bst_lb = lb;
            cnt_step = 0;
        else
            cnt_step = cnt_step + 1; % 否则下界未更新计数+1
        end

        % 更新乘子
        [lr_case.mu, cnt_step, lr_para.alpha] = update_mu(lr_case, lr_para, lb, bst_ub, lb_sqc, lb_loc, cnt_step);

        % 迭代记录
        rec_lb(iter)  = lb;         % 下界记录 record
        rec_ub(iter)  = bst_ub;     % 上界记录
        

        % 打印
        gap = (bst_ub-bst_lb) / bst_ub; % 计算gap
        t = toc;
        if lr_para.print
            formatSpec = "iter:%.0f, best-ub:%.2f, best-lb:%.2f, gap:%.4f%%, time:%.2f, ub:%.2f, lb:%.2f\n";
            fprintf(formatSpec, iter, bst_ub, bst_lb, gap*100, t, ub, lb);
        end
        

        % 快速慢速切换
        if gap <= lr_para.dfs_gap && ~flag_modify % gap小于一定的值 关闭快速模式
            flag_fast = false;  % 关闭快速模式
            flag_modify = true; % 强制修正
            % 快速模式获得的下界不是最优解,因此下界是虚标的,在此处进行修正
            [lb_val_x, ~] = lb_x(lr_case, flag_fast);  % 获取Sub_2的值
            [lb_val_y, ~] = lb_y(lr_case);             % 获取Sub_1的值
            bst_lb = sum(lb_val_x) + sum(lb_val_y);    % 强制修正下界
            gap = (bst_ub-bst_lb) / bst_ub;            % 修正之后的gap
        end

        % 终止条件
        if t > lr_para.tau_lim
            disp('stop, time') % 时间限制
            cnt_iter = iter;
            break
        else
            if lr_para.alpha < lr_para.alpha_min
                disp('stop, alpha') % 迭代值限制
                cnt_iter = iter;
                break
            elseif gap < lr_para.xi
                disp('stop, gap') % gap值达标
                cnt_iter = iter;
                break
            end
        end
    end
    if cnt_iter == lr_para.eta_lr
        disp('stop, iteration') % 迭代次数限制
    end

    %% 封装 & 返回      
    % 重新计算
    t = toc;
    lr_result         = struct();
    lr_result.rec_lb  = rec_lb;
    lr_result.rec_ub  = rec_ub;
    lr_result.rec_ils = rec_ils;
    lr_result.bst_loc = bst_loc;
    lr_result.bst_sqc = bst_sqc;
    lr_result.bst_ub  = bst_ub;
    lr_result.bst_lb  = bst_lb;
    lr_result.iter    = cnt_iter;
    lr_result.time    = t;


    end

    function [best_nb_ub, best_nb_loc, current_ub, current_loc] = ub_ils(lr_case, ub, location, lr_para)
    %UB_ILS 局部迭代搜索
    %% 初始化
    best_nb_ub = ub;                 % 记录全局最佳上界
    best_nb_loc = location;          % 记录全局最佳选址方案

    current_ub = ub;            % 当前上界
    current_loc = location;     % 当前方案

    temperature = -(lr_para.theta_sa-1)*ub/log(0.5);  % 初始温度
    t_gap = temperature / (lr_para.eta_ils*0.4);      % 每次降低的温度

    for iter = 1:lr_para.eta_ils
        % 求邻域
        nb = get_nb(current_loc); % 获取邻域
        nb_sz = size(nb,1);

        % 求所有邻域中成本最小的几个
        nb_cost_rec = zeros(nb_sz,1);
        for i = 1:size(nb,1)
            [nb_cost_rec(i), ~] = ub_xy(lr_case, nb(i,:), true); % 求快速解
        end
        [~, sort_ind] = sort(nb_cost_rec);    % 最小成本索引

        % 计算前 5 个 最佳索引的精准上界
        temp_best_loc = nb(sort_ind(1:5),:);
        temp_cost_rec = zeros(5,1);
        for i = 1:5
            [temp_cost_rec(i), ~] = ub_xy(lr_case, temp_best_loc(i,:), false); % 求快速解
        end
        
        [best_nb_cost, min_ind] = min(temp_cost_rec);

        % 接受上界
        if best_nb_cost < current_ub    % 得到更优质的解
            current_ub = best_nb_cost;              % 记录临时上界
            current_loc = temp_best_loc(min_ind,:); % 临时选址方案
        else % 概率接受不那么好的解
            if rand < exp((best_nb_cost - current_ub )/temperature)
                current_ub = best_nb_cost;          % 记录临时上界
                current_loc = nb(i,:);          % 临时选址方案
            end
        end
        
        % 更新最优解
        if current_ub < best_nb_ub % 获得全局最优解直接返回
            best_nb_ub = current_ub;             % 记录全局最佳上界
            best_nb_loc = current_loc;           % 记录全局最佳选址方案
        end
        
        % 更新温度
        temperature = temperature - t_gap;
        if temperature <= 0
            temperature = 0.0001;
        end
    end

    end

    function ns = get_nb(location)
    location(1) = []; % 先删除虚拟设施
    opened = find(location == 1);
    closed = find(location == 0);

    % 计算需要邻域的大小
    ns_sz = length(opened) + length(closed) + length(opened) * length(closed);
    ns = false(ns_sz,length(location));

    count = 1;
    % 关闭一个设施
    for i = 1:length(opened)
        temp = location;
        temp(opened(i)) = 0;
        ns(count,:) = temp;
        count = count + 1;
    end

    % 开启一个设施
    for i = 1:length(closed)
        temp = location;
        temp(closed(i)) = 1;
        ns(count,:) = temp;
        count = count + 1;
    end

    % 交换设施状态
    for i = 1:length(opened)
        for j = 1:length(closed)
            temp = location;
            temp(opened(i)) = 0;
            temp(closed(j)) = 1;
            ns(count,:) = temp;
            count = count + 1;
        end
    end

    % 补充虚拟设施
    ns = [true(size(ns,1),1) ns];
    end


    function [mu, cnt_step, alpha] = update_mu(lr_case, lr_para, lb, ub, plan_lb, location, cnt_step)
    %乘子更新

    % 提取
    I = lr_case.I;
    bar_J = lr_case.bar_J;
    mu = lr_case.mu;
    alpha = lr_para.alpha;

    % 更新系数
    if cnt_step > lr_para.kappa_lb % 长时间未更新下界则步长因子打折
        alpha = alpha / lr_para.theta_lr; % 步长系数打折
        cnt_step = 0; % 打折后计数器归零
    end

    temp = zeros(length(I), length(bar_J)); %记录客户i是否使用了设施j
    % 计算分母
    for i = 1:size(plan_lb,1)
        fac_use = plan_lb(i, 2:end); % 对于i来说使用了的设施
        fac_use(fac_use==0) = [];
        temp(i, fac_use) = 1;
    end
    violate = temp - location; % 违背约束

    % 计算迭代步长
    k_step = lr_para.alpha * (ub-lb) / sum(abs(violate),"all");

    % 更新乘子
    for m =2:length(bar_J)
        j = bar_J(m);   % j 是设施
        if location(j)
            y_j = 1;
        else
            y_j = 0;
        end

        for n = 1:length(I)
            i = I(n) - bar_J(end); % i 是客户
            if any(plan_lb(i,2:end) == j)
                x_ikj = 1;
            else
                x_ikj = 0;
            end
            mu(i, j) = mu(i, j) + k_step * (x_ikj - y_j);
            if mu(i,j) < 0
                mu(i,j) = 0;
            end
        end
    end


    function [obj, plan, trans_cost] = ub_xy(lr_case, location, flag_fast)
    %UB_YX求解模型的上界
    % OBJ 返回原问题的最优目标函数
    % plan 返回最优路径方案
    % trans_cost 返回各路径的运输成本

    %% 初始化
    % 提取
    I = lr_case.I;
    bar_J = lr_case.bar_J;
    data = lr_case.data;
    q = lr_case.q;
    max_try = lr_case.max_try;
    dmd = data.dmd;
    price = data.price;

    trans_cost = zeros(length(I),1);    % 记录每个客户的期望运输成本
    fix_cost = data.fix(location);      % 计算固定成本
    plan = zeros(length(I), max_try+1); % 记录每个客户的方案
    location(1) = 1;                    % 虚拟设施指定是已经建设的
    q_loc = q(location);                % 已建设施的损坏概率
    located = find(location==1);        % 已经建设的设施的索引

    price_located = price(location,:);  % 建设的节点之间的价格
    price_cus_fac = price(length(bar_J)+1:end,:); % 顾客和建设的节点之间的距离
    parfor i = 1:length(I)
        cus = I(i);
        % 初始化客户相关变量
        
        dmd_cus = dmd(i); % 节点需求
        price_cus = [price_located; 
                    price_cus_fac(i,:)]; % 价格矩阵去掉多余行
        price_cus = price_cus(:,location);  % 价格矩阵去掉多余列

        % dijkstra
        [pind_without_cus, trans_cost_cus] = mod_dijkstra(price_cus, q_loc, dmd_cus, max_try);
        
        % 记录
        trans_cost(i) = trans_cost_cus;
        temp = [cus, pind_without_cus];

        if length(temp) < max_try+1
            plan(i,:) = [temp, zeros(1, max_try+1-length(temp))];
        else
            plan(i,:) = temp; % 记录方案
        end
    end

    for i = 1:size(plan,1)
        pind_with_cus = plan(i,:);
        pind_with_cus(pind_with_cus==0) = [];
        temp = [pind_with_cus(1), located(pind_with_cus(2:end))];
        if length(temp) < max_try+1
            plan(i,:) = [temp, zeros(1, max_try+1-length(temp))];
        else
            plan(i,:) = temp; % 记录方案
        end
    end

    obj = sum(trans_cost) + sum(fix_cost); % 目标函数值

    if ~flag_fast
        % 快速模式不启动dfs
        [obj, plan, trans_cost] = ub_dfs(I, bar_J, location, plan, trans_cost, data, q, max_try);
    end

    end


    function [pind_without_cus, trans_cost] = mod_dijkstra(price, q_loc, dmd_cus, max_try)
    %MOD_DIJKSTRA 修正的Dijkstra算法
    % 输入一个i行j列的距离矩阵(从i到j),返回从最后一个点到第一个点最短成本的路径
    % 传入的距离矩阵最后一行表示客户,第一行是虚拟设施

    start = size(price,1);  % 初始起点
    fee = 0;                % 当前产生的费用
    preceding_ind = zeros(1, length(q_loc));   % 前向记录
    prob = ones(1, length(q_loc)+1);           % 概率记录
    unmark = 1:length(q_loc);                  % 未标记的点
    weight = inf * ones(1, length(q_loc));     % 点的权重

    count = 0;
    while count < max_try
        % 更新未扫描的节点
        for j = 1:length(unmark)
            % 上一个点产生的费用 加上 到这个点需要的费用
            temp = fee + prob(start) * dmd_cus * price(start, unmark(j));

            if temp < weight(unmark(j)) % 得到新的权重
                weight(unmark(j)) = temp;
                preceding_ind(unmark(j)) = start; 
            end
        end

        [fee, mind_unmark] = min(weight(unmark));   % 在所有未标记的权重中选取最小
        start = unmark(mind_unmark);                % 更新起点
        unmark(unmark==start) = [];                 % 更新标记
        prob(start) = prob(preceding_ind(start)) * q_loc(start);   % 更新概率
        
        if start == 1
            break
        end
        if isempty(unmark)
            break
        end
        count = count + 1;
    end

    trans_cost = weight(1);
    pind_without_cus = get_plan(preceding_ind, max_try, 1, size(price,1));

    end

    function plan_without_cus = get_plan(preceding, times, p_start, p_end)
    % 返回一个不带客户的路线方案
    % 给定DIJKSTRA算法的跟踪索引,返回一个路径
    plan_without_cus = zeros(1, times);
    ind = p_start;
    plan_without_cus(1) = ind;

    count = 2;
    while 1
        if preceding(ind) == p_end
            break
        end

        plan_without_cus(count) = preceding(ind);
        ind = preceding(ind);

        count = count+1;
    end
    plan_without_cus = plan_without_cus(end:-1:1); % 逆转
    plan_without_cus(plan_without_cus==0) = [];    % 除0
    end

    function [lr_ub, plan, trans_cost] = ub_dfs(I, bar_J, location, plan, trans_cost, data, q, max_try)
    %UB_DFS 上界DFS搜索
    pi = data.price(2,1);    % 惩罚成本
    cus_mu = zeros(1,length(bar_J)); % 上界没有乘子
    dmd = data.dmd;
    price = data.price;
    parfor j = 1:length(I)
        cus = I(j);
        best_r = plan(j,:); % 当前最优路径

        ub = trans_cost(j); % 当前上界
        cus_dmd = dmd(j); % 客户的需求
        
        fac = find(location==1); % 已经建设的设施

        R = coder.ignoreConst(max_try-1);
        coder.varsize('cus', [1 100], [0 1]);
        coder.varsize('best_r');
        coder.varsize('bar_J');
        coder.varsize('price');
        coder.varsize('q');
        coder.varsize('cus_mu');

        [ub, best_r] = lb_dfs(cus, best_r, ub, 0, 1, fac, cus_dmd, price, R, pi, q, cus_mu); % 深度优先搜索
        trans_cost(j) = ub; % 更新上界
        plan(j,:) = 0;
        if length(best_r) < max_try+1
            plan(j,:) = [best_r, zeros(1, max_try+1-length(best_r))];
        else
            plan(j,:) = best_r; % 记录方案
        end
    end

    fix_cost = data.fix(location);      % 计算固定成本
    lr_ub = sum(trans_cost) + sum(fix_cost); % 上界
    end
    
    \end{lstlisting}
}

\newpage
\subsection*{Gurobi建模源码(Python)}
\vspace{-20pt}
\begin{lstlisting}[language=Python,
    breaklines=true, % 自动换行
    frame = leftline,
    linewidth = \textwidth,
    texcl = false,
    columns=flexible,
    lineskip=1pt,
    keywordstyle=\color{blue!90}\bfseries, %代码关键字的颜色为蓝色,粗体
    numbers=left,%左侧显示行号
    numberstyle=\small, %行号字体用小号
    showspaces=false, %
    fontadjust,
    basicstyle = \fontspec{Times New Roman},
    basicstyle = \songti
    ]
import gurobipy as gp
import numpy as np
import math
import copy
import os
import sys


currentPath = os.getcwd().replace('\\','/')    # 获取当前路径
print(currentPath)


def previous(i, j, J):
before = copy.deepcopy(J)
before = [i] + before
if j != 0:
    before.remove(j)
return before


def later(i, j, bar_J):
    after = copy.deepcopy(bar_J)
    if i != j:
        after.remove(j)
    return after

def get_sol(x, I, bar_J, max_visit_num):
    plan = np.zeros((len(I), max_visit_num+1))
    for i in I:
        plan[i-I[0], 0] = i
        for k in later(i,i,bar_J):
            if math.isclose(x[i,i,k].X, 1, rel_tol=0.05):
                plan[i-I[0], 1] = k
                ind = k
                break
        
        count = 1
        flag = False
        while True:
            for k in later(i, ind, bar_J):
                if ind == bar_J[0]:
                    flag = True
                    break
                if math.isclose(x[i,ind,k].X, 1, rel_tol=0.05):
                    count = count+1
                    plan[i-I[0], count] = k
                    ind = k
            
            if flag:
                break
    return plan

# 导入数据
np.set_printoptions(suppress=True)    # 取消numpy打印的科学计数法


# cost 矩阵的第一索引位置是0 默认为虚拟设施
root = '../data/SnyderData/49nodes/'
cost = np.loadtxt(root+'cost.csv',  # 相对路径下的csv文件
                    dtype=None,         # 数据类型默认
                    encoding='UTF-8',   # 注意此文件为UTF-8格式且取消BOM
                    delimiter=',')      # 分隔符

dmd = np.loadtxt(root+'dmd.csv',  # 相对路径下的csv文件
                    dtype=None,         # 数据类型默认
                    encoding='UTF-8',   # 注意此文件为UTF-8格式且取消BOM
                    delimiter=',')      # 分隔符

fc = np.loadtxt(root+'fc.csv',  # 相对路径下的csv文件
                    dtype=None,         # 数据类型默认
                    encoding='UTF-8',   # 注意此文件为UTF-8格式且取消BOM
                    delimiter=',')      # 分隔符

                      
rho = 0.05 # 损坏概率参数
max_visit_num = 5 # 客户的最大尝试次数

q = rho * np.exp(-fc/200000) # 损坏概率
q[0] = 1

cus_num = len(dmd) # 客户数
node_num = cost.shape[0] # 节点数(虚拟 设施 客户)
fac_num = node_num - cus_num - 1 # 设施数

# 集合设置
J = [j for j in range(1, fac_num+1)] # 设施集合
I = [i for i in range(fac_num+1, node_num)] # 客户集合
bar_J = [j for j in range(0, fac_num+1)] # 设施拓展集合
R = [r for r in range(1,max_visit_num)] # 等级

# 常数集合
lmd = {i : dmd[i-fac_num-1] for i in I} # lambda需求
c = {(i,j) : cost[i,j] for i in bar_J+I for j in bar_J} # 价格
f = {j : fc[j] for j in J} # 建设成本

m = gp.Model()

y = m.addVars((j for j in J), vtype = gp.GRB.BINARY, name = 'y')
x = m.addVars(((i, j, j_p) for i in I for j in J+[i] for j_p in later(i,j,bar_J)), vtype = gp.GRB.BINARY,name = 'x') # 弧(j,j_p)属于客户i
p = m.addVars(((i, j, j_p) for i in I for j in J+[i] for j_p in later(i,j,bar_J)), lb = 0, ub = 1, vtype = gp.GRB.CONTINUOUS,name = 'p') # 属于客户i的弧(j,j_p)的概率
w = m.addVars(((i, j, j_p) for i in I for j in J+[i] for j_p in later(i,j,bar_J)), lb = 0, ub = 1, vtype = gp.GRB.CONTINUOUS,name = 'w') # 期望价格

item_1 = gp.quicksum(f[j] * y[j] for j in J)
item_2 = gp.quicksum((lmd[i] * c[k,j] * w[i,k,j]) for i in I for j in bar_J for k in previous(i,j,J))
m.setObjective(item_1 + item_2)

m.addConstrs((gp.quicksum(x[i,k,j] for k in previous(i,j,J)) <= y[j] for i in I for j in J), name='assign2open')
m.addConstrs((gp.quicksum(x[i,i,j] for j in later(i,i,bar_J)) == 1 for i in I), name = 'flowin')
m.addConstrs((gp.quicksum(x[i,j,0] for j in previous(i,0,J)) == 1 for i in I), name = 'flowout')

m.addConstrs((gp.quicksum(x[i,j,k] for k in later(i,j,bar_J)) 
           == gp.quicksum(x[i,k,j] for k in previous(i,j,J)) for i in I for j in J), name = 'balance')

m.addConstrs((p[i,i,j] == x[i,i,j] for i in I for j in later(i,i,bar_J)), name = 'probinit')

m.addConstrs(((q[j] * gp.quicksum(w[i,k,j] for k in previous(i,j,J)) 
== p[i,j,j_p]) for i in I for j in J for j_p in later(i,j,bar_J)), name = 'probbalance')

m.addConstrs(((gp.quicksum(x[i,j,k] for j in J+[i] for k in later(i,j,bar_J))) <= max_visit_num for i in I), name = 'maxtry')

m.addConstrs((w[i,j,k] <= p[i,j,k] for i in I for j in J+[i] for k in later(i,j,bar_J)), name='u1b')
m.addConstrs((w[i,j,k] <= x[i,j,k] for i in I for j in J+[i] for k in later(i,j,bar_J)), name='u2b')
m.addConstrs((w[i,j,k] >= p[i,j,k] + x[i,j,k] -1 for i in I for j in J+[i] for k in later(i,j,bar_J)), name='lb')


m.Params.MIPGap = 0.000003
m.Params.timeLimit = 1000


# m.Params.LogFile =  "SolvingLog.log"
# m.write('Model.lp')


m.optimize()


# m.computeIIS()
# m.write('Model.ilp')
# m.write('Model.lp')


# m.write('Solution.sol')

print('求解完成')

\end{lstlisting}


