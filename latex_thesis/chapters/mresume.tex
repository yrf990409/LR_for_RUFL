%\mmchapter{作者简历}
 \setlength{\baselineskip}{16pt}
\chapter{作者简历及攻读硕士学位期间取得的研究成果}\zihao{5}
\setlength{\parindent}{0pt}

%[内容采用五号宋体]  包括教育经历、工作经历、攻读学位期间发表的论文和完成的工作等。行距16磅,段前后各为0磅。

一、作者简历

2021年,毕业于北京交通大学,交通运输学院,物流工程专业,获工学学士学位;

2021年至2023年,就读于北京交通大学,交通运输学院,交通运输专业,导师员丽芬副教授;

\vspace{10pt}
二、发表论文

[1] \textbf{RUNFENG YU}, LIFEN YUN, CHEN CHEN, YUANJIE TANG, HONGQIANG FAN, YI QIN, Vehicle Routing Optimization for Vaccine Distribution Considering Reducing Energy Consumption [J]. Sustainability, 2023, 15(2): 1252.(SCI)

[2] \textbf{RUNFENG YU}, LIFEN YUN, HONGQINAG FAN, YANXI LIU, MINYU JIN. Optimization of Vehicle Routing Problem for Vaccine Distribution [C], Washington DC, United States: Transportation Research Board 101st Annual Meeting, 2022.(会议)

[3] \textbf{RUNFENG YU}, LIFEN YUN, HONGQINAG FAN, YANXI LIU, MINYU JIN. Optimization of Vehicle Routing Problem for Vaccine Distribution [C], 武汉: 世界交通大会, 2022.(会议) 


[4] 员丽芬, \textbf{余润峰}, 范宏强, 张蜇. 不完美信息下考虑设施独立损坏概率的固定 费用选址模型及算法 [C], 中国物流学术年会论文, 2022.(会议)

[5] YANXI LIU, LIFEN YUN, HONGQINAG FAN, \textbf{RUNFENG YU}, MINYU JIN. Location-Routing Problem of Pharmaceutical Facilities with Soft Time Window [C], Washington DC, United States: Transportation Research Board 101st Annual Meeting, 2022.(会议)

\vspace{10pt}
三、专利

[1] 员丽芬, 张钰儒, 余润峰. 仓店一体模式下前置仓选址与路径联合优化方法: CN202210529946 [P]. 2022.08.23.

%\section*{访学经历}
%2011年8月至2011年11月,访问XX大学XX系,合作导师:XX教授;
%
%\section*{承担的科学研究工作}
%  (1) 主持,XX项目...。
%
%  (2) 参加,...;
%

\vspace{10pt}
四、获奖情况

(1) 一等奖学金,北京交通大学,2022

(2) 神州控股校园极客大赛(全国第一名),神州控股,2022

(3) 日日顺创客训练营(银奖),青岛日日顺,2022

