 \setlength{\baselineskip}{20pt}
\begin{englishabstract}
	
    Logistics nodes are important support for logistics networks, 
    and their scientific layout and stable operation are key 
    factors that restrict network reliability. 
    Therefore, it is practical to consider the possibility of logistics node 
    disruption at the beginning of logistics network design. 
    This can not only improve the reliability of logistics networks to cope 
    with increasingly frequent natural disasters and human accidents, 
    but also reduce operating costs and increase network operational efficiency 
    to promote high-quality development of the logistics industry.

    The purpose of this study is to strengthen the reliability of logistics 
    networks and reduce the high costs of temporary decision-making when nodes fail. 
    Taking into account the impact of node disruption risk on logistics networks, 
    this article constructs a mathematical model for this 
    problem, focusing on node disruption conditions and corresponding 
    countermeasures. The objective of the model is to find the
    location of logistics nodes with the minimum expected total cost and customer 
    distribution plan. This paper proposes a linearization method 
    and conducts a series of property analyses on the model.
    
    Since the problem is NP-hard, 
    in order to solve large-scale instances, this paper develops a 
    Lagrangian relaxation heuristic with iterative local search operator. 
    The optimal value of the relaxed problem provides a lower bound for the 
    original problem, the heuristic method obtains an upper bound, 
    and local iterative search can further enhance the upper bound.
    
    This paper designs a series of numerical experiments to demonstrate the 
    performance of the above algorithm in solving large-scale problems. 
    The experimental results show that the Lagrangian relaxation is 
    an effective method for solving such location problems, 
    with strong convergence and high quality lower bounds, 
    and is less affected by instance parameters. 
    The topological structure of a reliable logistics network in that region is obtained. 
    The sensitivity analysis results reveal relevant economic laws, 
    which indicate that the probability of node disruption is a relatively important 
    factor affecting network reliability, and that relying solely on increasing the 
    number of backup nodes for each customer has limited effect on improving network 
    reliability.
    
    % The above research work not only enriches the theory of logistics node 
    % location, but also extends related algorithm research, 
    % providing theoretical support for logistics enterprises to carry out 
    % location planning and stabilize logistics networks.


\noindent\englishkeywords{Logistics Node Location; Facility Location Problem; Lagrange Relaxation; Iterative Local Search}
\end{englishabstract}