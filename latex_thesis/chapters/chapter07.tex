 \setlength{\baselineskip}{20pt}
\chapter{总结与展望}
\label{cha:summary}

本章的\ref{sec:summary}节总结本文工作内容,
并指出了本文可完善的部分,
为后续工作指明改进方向,
此外,
第\ref{sec:future}节还展望了有关可靠选址问题的未来研究方向,
提出了当前研究的热点和未来有意义的研究内容。

% TODO 拉格朗日松弛-迭代局部搜索算法 

\section{工作总结}
\label{sec:summary}
本文研究了物流节点可靠选址及客户分配模型与算法,
研究的主要内容为模型和算法。
考虑到物流节点在现实运营中将面临一系列可能导致其失效的风险因素,
以及节点失效时客户的行为,
本文构建了考虑节点失效风险的物流网络选址模型,
并设计了基于迭代局部搜索改进的拉格朗日松弛算法。
数值实验结果展示了算法可求解大规模算例的优越性能。
最后,本文进行物流节点可靠选址算例计算,
为物流节点选址提供了理论支撑。
论文的主要研究内容总结和结论如下:

\begin{enumerate}[label=(\arabic*),leftmargin=0pt,itemindent=3.5\ccwd, nosep]
    \item 基于可靠选址理论,考虑到客户的不完全信息状态,
    本文描述了客户的试错过程,
    构建了节点端提供服务的物流节点可靠选址问题,
    该问题被证明为NP-hard问题。
    本文的问题可构建为非线性整数规划模型,
    使用线性化技术消除模型中的非线性成分。
    此外,本文提出了一个与该问题相关的最短路问题,
    并给出了该问题的数学模型。
    
    \item 本文定制了基于迭代局部搜索改进的拉格朗日松弛算法,
    启发式算子快速获得模型的上下界的近似值,
    深度优先搜索算法再对上下界结果进行改进,得到上下界的精确值。
    迭代局部搜索算子对拉格朗日松弛上界选址方案进行局部搜索以进一步提升上界质量。

    \item 本文对比了基于迭代局部搜索改进的拉格朗日松弛算法和求解器的性能,
    实验结果表明基于迭代局部搜索改进的拉格朗日松弛算法具有求解大规模问题和现实案例的能力。
    本文详细讨论了算法每个算子的效果,
    通过数值实验对比了算子的效率,
    虽然深度优先搜索算子是非多项式时间算法,
    但在求解该问题时仍展现了良好的效果。
    
    \item 本文进行了灵敏度分析,
    结果表明:
    每个客户拥有一个常用节点和两个备用节点即可提升网络的可靠性。
    在一定范围内,
    随着运价和失效概率的提升,
    系统的总成本提升但能保持选址方案不变。
    当运价和失效概率的提升超过一定幅度,
    增加节点建设数量是降低系统成本、维持系统可靠性的方法之一。
    
\end{enumerate}

限于作者的自身理论水平以及代码编写水平,
本文的研究内容也有一系列的局限性。
模型方面,本文描述了客户常用节点失效后的试错以寻找设施的行为,
绝对的不完全信息场景和绝对的完全信息场景都是假设前提,
但现实中的情况远远比实际情况复杂,
很可能是不完全信息场景和完全信息场景的混合。
此外,模型假设每个节点的失效是彼此独立的,
该假设主动忽视了节点之间的联系。
算法方面,虽然拉格朗日松弛算法能提供一个相对较紧的下界,
但对于大规模的问题,
该算法的效果同样有限。
并且,拉格朗日松弛算法在大多数情况下只能得到一个近似最优解。
此外,算法的算子为非多项式时间算法,
对于一些特殊且复杂的问题,
在最差的情况下无法保证求解时间。
最后,受限于作者编写代码水平和Matlab解释型语言的性能,
当前算法仍有进一步改进的空间。


\section{研究展望}
\label{sec:future}

针对当前研究的局限性,
未来研究的可改进之处以及研究方向如下:

\begin{enumerate}[label=(\arabic*),leftmargin=0pt,itemindent=3.5\ccwd,listparindent = 2\ccwd, nosep]
    \item 现有模型考虑因素的改进。
    
    未来研究可对当前的模型考虑的不完全信息场景进一步改进,
    可混合完全信息和不完全信息场景,
    使得模型更加贴近现实。
    此外,现实中风险可导致多个节点同时失效,
    模型的节点失效独立性假设也可进一步改进。
    如何构建节点之间的失效关联方程,
    特别是离散选址模型中该方程的定义,
    是未来研究的重点。
    对于当前研究的可靠选址问题,
    未来的研究中还可将其应用至更多场景中,
    例如可靠基础设施选址(电动车充电站、零售企业选址)等,
    还可与其他经典选址问题融合,
    构建可靠选址-路径问题、可靠选址-库存问题、多周期可靠选址问题等。

    \item 现有算法的改进。
    
    在本文原模型的上界和下界的求解过程中,
    都不可避免地反复求解一个最短路问题。
    该最短路问题考虑了每个节点的失效可能,
    具有一定的特殊性,
    使得经典求解最短路问题的算法失效。
    但是,在该最短路问题中,
    所有弧的权重为正值,
    利用该特性可以降低求解该问题的复杂性。
    由于该问题同样为NP-hard问题,
    因此求解该问题的不存在多项式时间算法,
    但可能存在伪多项式时间算法。
    可考虑将脉冲算法、标签算法移植到本问题中,
    推动求解原问题的进程,
    使得算法更具有效率。

    \item 求解模型的其他算法。
    
    使用拉格朗日松弛算法得到结果不能保证其最优性,
    但可获得解的下界。
    利用本文中构建迭代局部搜索算子的思路,
    可以开发出元启发式算法。
    但元启发式算法只能给出一个可行解,
    并不能证明解的质量。
    所以,未来研究的另一个方向是开发求解模型的精确算法。
    Benders分解是一种求解混合整数规划问题、选址问题常用的精确算法。
    开发求解本问题的Benders分解算法,
    是未来研究中有深度的算法创新。

\end{enumerate}
