\setlength{\baselineskip}{20pt}
\begin{abstract}

    物畅其流则百业兴,我国正在加速构建以国内大循环为主体,
    国内国际双循环相互促进的新发展格局。
    打造安全可靠、高效便捷的现代物流网络系统迫在眉睫。
    物流节点作为物流网络的重要支撑,
    其科学布局、稳定运行是制约网络可靠性的重要关键因素。
    因此,在物流网络设计之初将物流节点失效的可能性考虑在内具有现实意义:
    不仅可以提升物流网络的可靠性以应对愈加频繁的自然灾害和人为事故,
    亦可以降低运营成本并增加网络运营效率以促进物流行业高质量发展。
    
    本文的研究目的是强化物流网络的可靠性,
    降低节点失效时因临时决策产生的高昂成本。
    综合考虑节点失效风险对物流网络的影响,
    针对可能存在的节点失效状况以及应对措施,
    为物流网络选址问题构建数学优化模型,
    目标是寻找期望总成本最小的物流节点建设位置和客户分配方案。
    本文提出线性化方法将非线性模型转换为线性模型,
    并对模型进行一系列的性质分析。
    
    由于考虑节点失效风险的物流网络选址问题是NP-hard问题,
    为求解该问题的大规模实例,
    本文开发了基于迭代局部搜索改进的拉格朗日松弛启发式算法。
    其中,
    松弛问题的最优值提供原问题的下界,
    启发式方法获得原问题的上界,
    局部迭代搜索可进一步强化上界。
    
    数值实验证明了上述算法求解大规模问题的性能。
    实验结果表明,
    拉格朗日松弛算法是求解该选址问题的有效方法,
    其收敛性强、下界质量高,受算例参数的影响较小。
    本文应用上述模型和算法得到可靠物流网络的拓扑结构。
    参数灵敏度结果揭示了相关经济规律,
    表明节点的失效概率是影响网络可靠性相对重要的因素,
    而仅依靠增加每个客户备用节点数量对于提升网络可靠性的效果有限。

\noindent\keywords{物流节点选址;选址问题;拉格朗日松弛;迭代局部搜索}
\end{abstract}