 \setlength{\baselineskip}{20pt}
\chapter{文献综述}
\label{cha:review}

选址问题是一个经典的优化问题。
该问题可追溯至Alfred Weber\cite{weber}于1909年首次提出的一个完善且成熟的工业选址理论,
该理论试图以最低成本找到工厂的最佳位置。
经过一个世纪的发展,已形成了一系列的经典选址模型。
基于需求和设施空间分布,
这些选址模型可分为解析选址模型、连续选址模型、网络选址模型和离散选址模型。
本文讨论的物流节点可靠选址问题属于离散选址问题,
而有关于上述四种模型的区别,
请参考Daskin的著作\cite{Daskin书}。
本章的第\ref{section:可靠选址概述}节回顾了可靠选址的发展历程,
总结了相关研究工作。
第\ref{section:节点选址概述}节描述了物流节点可靠选址理论研究现状,
指出当前研究的进展、考虑因素和解决方案。

\tinytodo[inline]{一般顺序应该是先写普通选址问题研究,
然后过渡到可靠选址理论研究;所以你的2.1和2.1 原因?
如果是后面针对物流节点选址,而2.1 是普通选址,可以从题目中进行区分。 
例如 普适性可靠选址理论综述,
物流节点选址理论综述。\newline 
\newline
\textbf{已修正}:\newline
\textbf{标题}:2.1 可靠选址理论综述 \& 2.2 物流节点可靠选址理论综述
第一段写的是可靠选址理论综述,
第二段写的是与物流相关的可靠选址综述。

\textbf{内容}:第一段由普通选址理论引出可靠选址理论。
}

\section{可靠选址理论综述}
\label{section:可靠选址概述}

\added[id=yrf]{
    离散选址问题是指
    给定一个网络拓扑结构,
    在设施候选建设位置集合中寻找满足条件的最佳子集。
    UFL问题和P-中值问题是离散选址问题的重要分支,
    二者的共同点是在网络中以最小的成本为设施选择建设的最佳位置并服务所有客户\cite{Daskin书},
    区别在于P-中值问题指定了设施选址的数量为P,
    而UFL选址问题对设施数量没有限制。
    Mak\cite{maxshen}及Daskin\cite{Daskin书}对P-中值问题和UFL选址问题的基础理论进行了详细阐述。
    本世纪初,学者们逐渐意识到不确定因素对设施网络的冲击和影响,开始着重研究节点选址中的不确定因素。
    考虑节点失效风险的选址问题起源于对供应链的中断现象的研究\cite{周愉峰综述},
    基于上述的UFL问题和P-中值问题,
    Snyder等人\cite{Snyder2005}提出了这两种选址理论的可靠版本:
    可靠P-中值问题(Reliable P-Median Problem, RPMP)和可靠无容量固定费用选址问题(RUFL)。
    可靠选址理论在经典选址模型的基础上,
    考虑每个设施损坏的概率,并为每个客户安排一个常用设施和多级备用设施,
    极大地提高了网络的可靠性。
    % 这种方法极大地提升网络的可靠性。
}

\deleted[id=yrf]{
    可靠选址理论研究是在Snyder等人\cite{Snyder2005}
    提出可靠P中值问题和可靠无容量限制设施选址问题之后开展的。
    该研究基于P-中值问题(P-median Problem, PMP)和UFL理论提出了这两种选址理论的可靠版本:
    可靠P-中值问题(Reliable P-Median Problem, RPMP)和可靠无容量固定费用选址问题(RUFL)。
    P-中值问题和无容量固定费用选址问题的共同点是在网络中以最小的成本为设施选址并服务所有客户\cite{Daskin书},
    区别在于P-中值问题指定了设施选址的数量为P,
    而UFL选址问题对设施数量没有限制。
    Mak\cite{maxshen}及Daskin\cite{Daskin书}对P-中值问题和UFL选址问题的基础理论进行了详细阐述。
    可靠选址问题的理论在经典选址模型的基础上,
    考虑每个设施损坏的概率,并为每个客户安排一个常用设施和多级备用设施。
}
Snyder\cite{Snyder2005}在其可靠选址基础理论中,
将设施分为完全可靠类型(即永久不会失效)和不完全可靠类型,
在后续研究中该理念仍被采纳\cite{rpmp,lim,李汉卿},
但大部分学者将所有设施都认定为不完全可靠状态,
即每个设施都有可能失效。
在Snyder\cite{Snyder2005}的研究基础上,
Cui\cite{Cui2010}、Shen\cite{Shen2011}、王艳敏\cite{王艳敏}松弛了设施失效概率相等的假设,
即设施的失效概率不同。
Aboolian\cite{Aboolian}不仅松弛了失效概率相等的假设,
还松弛了每个客户拥有的备用设施的数量上限,
但大部分学者仍接受客户拥有有限个备用设施的约束。

在可靠选址基本理论的基础上,
更多现实因素加入模型构建中。
首先,
导致设施损坏的灾害往往也使得通讯中断,
或者使得客户对设施的状态没有准确把握,
导致造成不完全信息状态的产生。
Berman\cite{BermanIncompleteInformation}最先研究了这种状态下的设施选址问题,
\replaced[id=syc]{Yun\cite{yun2015}基于RUFL理论提出了不完全信息下的可靠选址理论。
在该研究成果的基础上,松弛了节点失效概率相等的假设\cite{yun2017},
并考虑了返程的成本\cite{YUN2020},对所研究的可靠选址理论进行了完善。}
{Yun\cite{yun2015}基于RUFL理论提出了不完全信息下的可靠选址理论。
在Yun\cite{yun2015}的研究成果的基础上,
Yun\cite{yun2017}松弛了节点失效概率相等的假设,
Yun\cite{YUN2020}考虑了返程的成本。} 
考虑到设施可在失效之前被强化的可能,
Li等人\cite{qingwei,qingwei2,qingwei3}提出使用有限的成本强化设施,
以降低系统的总成本。
在Li\cite{qingwei2,qingwei}的研究中,
假设设施被强化后转变为完全可靠设施,
因此要求每个客户只能拥有一个常用设施和一个备用设施,
客户拥有完全可靠的设施后并不需要其余备用设施。
除了增加预算抵抗设施的失效,
风险规避也是提升可靠性的方法之一。
Yu\cite{risk_averse1,risk_averse2}提出了风险规避型可靠选址理论。
在以往的理论中,通常计算整个网络的最小期望成本,
Yu\cite{risk_averse1,risk_averse2}则控制每个客户承担的风险。
设施的失效可以是独立的,也可以是关联的,
该研究可以得到比传统模型更可靠的网络结构。

经典的RUFL理论考虑了设施的损坏失效情况,
但假设设施失效是独立的。
Lu等人\cite{lu_correlated}假定设施的失效之间存在一定的关联性,
采用了鲁棒优化方法研究了给定边际失效概率条件下的最小期望成本选址方案。
鲁棒优化是研究设施失效关联性的一种方法之一,
与鲁棒优化相关的可靠选址研究还可以参见Du等人\cite{Du}设计的双层可靠P-中心网络结构,
Peng等人\cite{peng}研究了考虑节点失效的物流网络结构,
Li等人\cite{Yongzhen}、An等人\cite{an2014reliable}提出了两阶段鲁棒优化可靠选址理论。
另一种研究失效相关的方法是连续近似模型,
Li\cite{Lixiapeng2010}构建了节点失效的关联性公式,
优化的目标是得到期望总成本最小的选址方案。
有关可靠选址的连续近似模型相关的讨论,
可见参考文献\cite{yun2019,fan,Cui2010}。
除了设施失效可能存在关联,
设施之间的相关性也可以体现在共同应对风险\cite{jiang},
使用更复杂的网络支撑结构提高整个网络的强度\cite{LiXP2013}。

上述的选址模型通过考虑设施失效的概率或边际失效概率以量化失效风险。
Berman\cite{BermanDist}将设施的可靠性与客户之间的距离关联,
研究了相关可靠选址问题。
此外,在现实中,不确定性还体现在多个方面,
例如需求不确定性的可靠设施选址\cite{cheng},
估计设施失效概率的误差\cite{lim2013facility}。
Snyder\cite{Snyder2006综述}概括了设施选址中的不确定因素,
Snyder\cite{SnyderReview}近期分析了供应链中的不确定因素,
Govindan\cite{Govindan}调查了供应链网络设计中的不确定因素。
这些研究表明,考虑不确定因素的影响是至关重要的,
可从供应链设计中借鉴经验并有效开展可靠设施选址研究。
周愉峰\cite{周愉峰综述}针对可靠选址理论的发展展开了详细描述,
并指出了未来的研究方向。

已知可靠选址问题是NP-hard问题\cite{Snyder2005},
这意味着当前不存在求解该问题的多项式时间算法。
因此,开发有效的求解算法是研究可靠选址的另一重点内容。
当前主流的可靠选址求解算法包含精确方法和启发式方法。
精确方法包括Branch-and-Bound, Benders分解,Branch-and-Price等方法,
启发式方法包括拉格朗日松弛
以及包括禁忌搜索、模拟退火、遗传算法等在内的元启发式方法。
通常,元启发式方法虽然不能保证求解的质量,
但可在可接受的时间内得到一个经过优化的可行解,
因此对于大规模问题,解决方法主要以启发式方法为主。

精确求解可靠选址问题的相关研究数量上相对较少,
求解方法主要以Benders分解算法为主。
有关Benders分解求解基本设施选址问题的研究可见参考文献\cite{fischetti,ortiz2019exact,wentges1996accelerating}。
Mohammad\cite{MohammadBenders}使用加速Benders分解算法求解了多层的可靠网络结构,
Azad\cite{Azad}开发了Benders分解算法解决了一个可应对节点失效的弹性供应链网络结构设计问题。
Pirniya\cite{Pirniya}使用了列生成(Colummn Generation)算法求解了可靠选址、客户分配及路径问题,
并取得了良好的效果。

而采用启发式算法求解可靠选址问题的研究相对较多。
拉格朗日松弛是一种求解整数规划问题以及混合整数规划问题的高效算法\cite{孙小玲,Hearn2009}。
对于求解最小化问题,
拉格朗日松弛不仅可以快速获得优化上界,
还提供了优于线性松弛的下界,
因此可以证明上界的质量。
拉格朗日松弛在求解选址问题方面取得了显著成就,
Geoffrion\cite{Geoffrion1974}首次使用拉格朗日松弛求解了UFL问题,
Cornuejols等人\cite{Cornuejols}首次使用拉格朗日松弛求解了P-中值问题。
拉格朗日松弛算法同样也是求解可靠设施选址问题的主流算法,
其设计过程具有灵活性,可根据问题定制。
因此即便都采用了拉格朗日松弛算法,
算法的技术细节都不尽相同。
Snyder\cite{Snyder2005}首次使用了拉格朗日松弛算法求解了RUFL问题,
后续的研究结果表明拉格朗日松弛求解RUFL问题效果显著:
Cui\cite{Cui2010}使用该算法求解了不等失效概率的可靠选址问题,
Yun\cite{yun2015}开发了拉格朗日松弛算法框架求解了不完美信息下的可靠选址问题,
Li\cite{qingwei}为考虑强化策略的可靠选址问题定制了拉格朗日松弛算法。
Yu\cite{risk_averse2}基于拉格朗日松弛框架设计了对偶分解算法求解了风险规避型可靠选址问题。

前文中提到拉格朗日松弛算法具有灵活性和实用性,
因此可将拉格朗日松弛作为算法框架,
将其他算法嵌入拉格朗日松弛算法中,
也可以将拉格朗日松弛嵌入其他算法中。
Xie\cite{xie}将列生成和局部搜索嵌入拉格朗日松弛,
研究了可靠选址路径问题。
Yu等人\cite{risk_averse1}将Branch-and-Cut算法和拉格朗日松弛算法相结合,
求解了风险规避型可靠选址问题。
此外,应用拉格朗日松弛算法求解可靠选址问题,
还可见参考文献\cite{YUN2020,qingwei2, wangjiguang},
有关拉格朗日松弛算法的介绍可见参考文献\cite{孙小玲},
有关算法原理的推导可见参考文献\cite{Fisher2004,Hearn2009}。

除了拉格朗日松弛之外,
还有学者采用其他启发式方法求解可靠选址问题,
包括Shen\cite{Shen2011}提出的近似算法和贪心算法,
Aboolian\cite{Aboolian}提出的近似方法,
以及Albareda\cite{rpmp}提出的近似流方法。
而元启发式方法也在求解大规模选址问题时展现了显著优势,
例如遗传算法\cite{Rahmani}、邻域搜索\cite{rainns}、禁忌搜索\cite{ts}、
变邻域搜索\cite{vns_ts}等。
求解可靠设施选址问题,
Peng等人\cite{peng}基于遗传算法和邻域搜索算法开发了混合元启发式方法,
Li\cite{qingwei3}开发了禁忌搜索算法,
Afify\cite{Afify}开发了进化学习算法,
王艳敏\cite{王艳敏}开发了模拟退火-粒子群算法。

综上,可靠选址问题的研究主要集中在两个方面:
其一是在构建模型时考虑各种现实因素,
以强化模型的现实意义。
从当前的研究成果来看,
大多可靠选址理论假设节点失效独立,
而有关节点失效相关联的研究成果较少。
在大多数可靠选址理论中,
节点的失效是以先验概率的形式表达的,
这对估计失效概率的精准程度有较高的要求,
有关失效概率预估误差对选址结果造成影响的研究较少。
其二是设计有效的算法求解该NP-hard问题。
以提升求解复杂模型的能力。
当前可靠选址求解算法研究的热门仍是启发式方法,
采用精确方法求解该问题的研究较为薄弱。


\section{物流节点可靠选址理论综述}
\label{section:节点选址概述}

物流节点是物流网络的构成要素,
节点的选址决策将显著影响物流系统的运行效率,
关系物流企业的经济利益。
窦志武等人\cite{窦志武}回顾了物流节点选址的方法,
包括重心法、整数规划模型法、层次分析法、数据包络分析法、模糊综合评价法等。
本文研究的物流节点可靠选址问题采用了整数规划模型法,
其他选址方法可见参考文献\cite{层次分析,重心法,数据包络分析}。

物流节点可靠选址的大部分研究关注于节点的状态,
特别是节点因某些原因不能提供服务的场景。
通常,物流节点失效状态由失效概率表示,
即该节点以某个概率值不能提供服务。
以供应链为研究对象,
王艳敏\cite{王艳敏}假设每个设施的失效概率不等,
建立了一个容量有限的供应链设施可靠选址的多目标模型;
张莹\cite{张莹}研究了供应中断风险的选址-路径问题,
考虑所有的设施均以先验概率发生失效,
多个设施也可以同时发生失效的场景;
汤罗浩\cite{汤罗浩}设计了一种可强化节点的供应链网络,
通过对节点进行保护并为需求点分配备份节点来提高供应链网络的可靠性;
王继光\cite{王继光}从供应链系统的可靠性和鲁棒性两方面研究了中断情境下的可靠选址问题,
针对随机中断和非随机中断分别提出了混合整数非线性规划模型和两阶段动态博弈模型。
以铁路建设为研究对象,
考虑了建筑材料的时效性需求,
周浩\cite{周浩}研究了铁路沿线的线状需求的物流节点连续型可靠选址问题,
获得了靠近需求线路中点的物流节点选址位置的解析解。
以区域物流为研究对象,
朱江华等人\cite{朱江华}以贵州省为案例开展物流园区可靠选址规划分析,
构建了考虑突发事件的物流园区选址与服务分配模型;
董鹏\cite{董鹏}以京津冀地区为例,
构建了信息失效情景下考虑双向行程的可靠节点连续选址模型。

而另一部分研究关注于线路的状态,
即节点和客户之间的线路发生中断从而不能向客户服务的场景。
颜高民\cite{颜高民}研究了突发事件下应急物流可靠路径搜索问题,
考虑了突发事件下路径畅通的不确定性,
构建了最短路和畅通路径综合模型,
并设计了一定规则修正路径。
李锐等人\cite{李锐}研究可靠绿色物流配送选址-路径问题的同时,
考虑了运输油耗和碳排放及配送中心和运输线路的中断,
建立物流配送网络选址-路径优化模型,
以最小总成本满足车辆路径约束。
考虑到供应网络中每条路径都存在运输中断风险,
任慧\cite{任慧}研究了多源协同供应的三级供应网络可靠选址路径集成优化问题,
在有限的设施(工厂、配送中心)能力下,
提出了运作成本和运输可靠性的双目标选址路径模型。
石褚巍等人\cite{石褚巍}研究了网络中存在节点和线路损坏不确定性的可靠物流网络,
提出了一种基于两阶段鲁棒优化的可靠物流网络设计方法。

除了节点和线路的不确定性,
需求和供应的不确定性也是研究的重点。
考虑到二次灾难可能造成应急配送中心失效以及救援过程中道路中断情况的发生,
张鹏阁\cite{张鹏阁}研究了需求不确定的应急物流选址-路径优化问题,
建立了一个包含常规型和可靠型的应急物流网络。
孙晓飞等人\cite{孙晓飞}研究了不确定环境下配送中心选址的多目标优化问题,
探讨了配送系统的可靠性。
姚琦\cite{姚琦}研究了生鲜农产品配送中心选址的问题,
考虑时间可靠度和品质可靠度对配送中心选址的影响,
运用层次分析法与模糊综合评价法建立了可靠物流配送中心选址模型。


此外,设施的失效常常与自然灾害相关联。
周愉峰\cite{周愉峰应急}等人以应急物资保障的及时性和可靠性为目标,
考虑在不同地区建立储备库的不同失灵概率,
构建了一个应急物资储备库的可靠P-中值选址模型,
并设计了一种拉格朗日松弛求解算法。
李政祥\cite{李政祥1}考虑设施的中断,
以成本和时间最小化为目标,
建立了应对灾害的应急物流多目标可靠选址模型。
李政祥\cite{李政祥2}构建了考虑设施设防及设施中断的多目标大规模地震灾害应急物资储备库的选址-分配模型,
设计了非支配解排序遗传算法(NSGA-II)以求解多目标问题。
朱建明\cite{朱建明}同时考虑设施可能的损毁情景以及设施两两之间的调度时间,
建立了可靠连通应急设施选址模型,
设计了基于遗传算法的求解方法。
于冬梅等人\cite{于冬梅}建立了中断情境下服务能力有限的可靠应急设施选址-分配多目标优化模型,
采用带精英策略的快速NSGA-II对模型予以求解。
王子墨\cite{王子墨}建立了考虑失效风险的救灾物资储备库可靠选址模型,
对储备库的位置、容量和需求点与救灾物资储备库的分配关系进行决策,
并通过遗传算法对模型进行求解。

综上,
物流节点的可靠选址研究将可靠选址理论与实际结合,
根据现实场景和需求
进一步拓展和应用可靠选址理论。
物流节点可靠选址问题的研究主要考虑了节点、线路、供应和需求的不确定性,
研究对象涵盖了供应链网络优化、物流园区设计、配送中心选址、应急物流规划等内容。


\section{本章小结}

选址问题是运筹优化领域经久不衰的一项重要研究内容,
而可靠选址研究是近二十年来选址理论的重要发展成果。
本章详细展开了可靠选址的发展过程,
介绍了可靠选址的起源以及可靠选址问题重要分支。
这些研究成果可分为理论创新和算法创新。
理论创新考虑了更多的现实因素,
而随着越来越多的现实因素加入模型中,
求解模型的困难程度提升,
也促进了相关算法的发展。

此外,可靠选址是物流节点选址的热门研究内容。
物流系统可能会面临各种不确定因素,
这些不确定因素可发生于供给端、需求端以及二者之间连接过程,
对物流网络造成严重的负面影响,
因此有必要在选址时将不确定因素考虑其中。
由本章文献综述可知,
考虑节点失效风险和不完全信息场景下客户试错策略的研究相对较少。
研究该问题有助于强化物流网络结构,
合理规划节点选址和客户分配,
为物流网络结构设计提供理论支撑。
为采用科学方法研究该问题,
本文的第\ref{cha:model}章构建了该问题的数学优化模型。







